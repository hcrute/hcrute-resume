% LaTeX file for resume 
% This file uses the resume document class (res.cls)

\documentclass{res} 
%\usepackage{helvetica} % uses helvetica postscript font (download helvetica.sty)
%\usepackage{newcent}   % uses new century schoolbook postscript font 
\newsectionwidth{0pt}  % So the text is not indented under section headings
\setlength{\textheight}{10.2in} % set text height big enough for box
\topmargin=-.5in       % to start box .5in from top of page
\oddsidemargin=-.6in   % to start box .5in from left of page
    
\begin{document}
 
%%%%%%%%%%%%%%%%%%%%%%%%%%%%%%%%%%%%%%%%%%%%%%%%%%%%%%%%%%%%%%%%%%%%%%%%%%%%
% The following lines define \boxaround, used to draw a box on the page.
% The parameter is the entire text of the resume. Must fit on one page!
%
% \boxaroundhmargin is the left & right margin around the text inside the box.
% \boxaroundvmargin is the top & bottom margin around the text inside the box.
% \boxrulethickness controls thickness of line used to draw the box.
% You can change these 3 things in the lines below:
%%%%%%%%%%%%%%%%%%%%%%%%%%%%%%%%%%%%%%%%%%%%%%%%%%%%%%%%%%%%%%%%%%%%%%%%%%%%%
\newdimen\boxrulethickness\newdimen\boxaroundhmargin\newdimen\boxaroundvmargin
\boxrulethickness=.5pt        %controls thickness of line 
\boxaroundhmargin=35pt        % about a half inch
\boxaroundvmargin=32pt        % to fit more text on page, make this smaller
%%%%%%%%%%%%%%%%%%%%%%%%% Don't read this stuff %%%%%%%%%%%%%%%%%%%%%%%%%%%%%%
\hsize=7.5in% \vsize=10.5in             % use bigger dimensions for box
\newbox\MACboxA  \newdimen\MACdimenA
% \borderandboxit is used inside \boxaround:
\def\borderandboxit#1#2#3{\vbox{\hrule height#2\hbox{\vrule width#2\hskip#1\hskip-#2%
  \vbox{\vskip#1\relax#3\vskip#1}\hskip#1\hskip-#2\vrule width#2}\hrule height#2}}
%
\long\def\boxaround#1{\vskip6pt
  {\MACdimenA=\hsize \advance\MACdimenA by-\boxaroundhmargin
   \advance\MACdimenA by-\boxaroundhmargin   % once for each side
   \setbox\MACboxA=\hbox to \hsize{\hskip\boxaroundhmargin%\hss
                     \vbox{\hsize=\MACdimenA
                           \vskip\boxaroundvmargin #1
                           \vskip\boxaroundvmargin}\hss}%
   \borderandboxit{0pt}\boxrulethickness{\box\MACboxA}}%
  \vskip2pt plus0pt minus0pt
}
%%%%%%%%%%%%%%%%%%%  End of \boxaround macro %%%%%%%%%%%%%%%%%%%%%%%%%%%%%%%%%
 
\boxaround{ % put the text on the page inside a box  

\name{HENRY CRUTE\\[12pt]}
\address{\bf Present Address\\122 Torrey Pine Terrace\\Santa Cruz, CA 95060\\
        (831) 325-8718} 
\address{\bf Permanent Address\\10440 Rock Creek Drive\\
         San Diego, CA 92131\\ (831) 325-8718}

 
\begin{resume}

\section{\bf  Experience}
\noindent\makebox[\textwidth]{\textit{Research Assistant at CITRIS} \hfill 6/14 - Present} \\
Managing and developing technical aspects of the SEAD Plug project (SEAD Systems). This includes working with a mentor, collaborating with students/professors, and working on a diverse internet of things system. Fixed and developed programs/protocols for firmware, front-end, back-end, and data analysis/signal processing.

\noindent\makebox[\textwidth]{\textit{Instructor at iD Tech Camps} \hfill 6/12 - 8/13} \\
Worked with a team of instructors and directors to run a camp at UCLA and UC Irvine. This includes working together by following a schedule, and completing tasks that rely on other instructors as well as teaching technical information. Taught an introduction to programming in Java using Eclipse, and LEGO Mindstorms Robotics.

\section{\bf  Education}
University of California Santa Cruz \\
B.S. Computer Engineering, expected June 2015 \\
G.P.A 3.57/4.0 (in major)

\section{\bf  Skills}
Very experienced with programming with Unix systems, and microcontroller systems. \\
Programming languages - C/C++, Java, Flex, Bison, Bash, Embedded C, Verilog, Assembly, Python, Matlab, HTML5/PHP/JavaScript/CSS, Perl, Scheme, Smalltalk, OCaml

\section{\bf  Example Projects}
\begin{itemize}
  \item FFTW Extended API - Built on top of the C fftw library. Abstraction from the main API, so it is easier to integrate into other C programs. Eliminates the hassle of memory management, and specifically written for the computation of forward and backwards one-dimensional fast fourier transform.
  \item Compiler - Built using C++, Flex, and Bison. The compiler scans, parses, type checks, and outputs appropriately translated assembly code. This language is a subset of C, and includes structs, loops, variables, if statements, scope, functions, and more.
	\item Pseudo NAT protocol - Built using C. This application emulates the regular Network Address Translation (NAT) protocol, and even uses multithreading to complete requests as fast as possible.
  \item AST Library - Built using C++. For my compiler during the parsing stage, it was required to create an abstract syntax tree where any node could have any number of children linked to it.
	\item Linked list library - Built using C. All memory allocation is taken care of in the library, and the main user of the functions doesn’t have to worry about anything. Is able to create nodes, link them in any order, and delete them.
\end{itemize}

\section{\bf  Honors}
Dean's Honors: Fall 2013 - Spring 2014
 
\section{\bf  Activities}
Tau Beta Pi Engineering Honor Society \\
Alpha Epsilon Pi Fraternity \\
Crown Student Activities Committee \\
Other interests include: Bicycling, Chess, Working Out, and Reading

\end{resume}

\vfill} %    end the material being boxed.
\end{document}
